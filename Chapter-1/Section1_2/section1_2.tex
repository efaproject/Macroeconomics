\documentclass[12pt,a4paper]{article}
\usepackage[left=1cm,right=1cm,top=2cm,bottom=2cm]{geometry}
\usepackage{amsmath} % Математические окружения AMS
 \usepackage{amsfonts} % Шрифты AMS
 \usepackage{amssymb} % Символы AMS
  \usepackage{mathtext} % Русские буквы в фомулах
  \usepackage{graphicx} % Вставить pdf- или png-файлы
\usepackage{multicol}
  \usepackage{euscript} % Красивый шрифт
  \usepackage{epsfig}
  \usepackage{epstopdf}
  
  \newcommand{\hypo}{\mathcal{H}}             % Символ гипотезы


 \usepackage[cp1251]{inputenc}% Кодировка исходного текста.
 \usepackage[russian]{babel} % Поддержка русского языка.
 \usepackage{indentfirst} % Отступ в первом абзаце.
\DeclareMathOperator{\rank}{rank}
 \newcommand*{\hm}[1]{#1\nobreak\discretionary{}%
            {\hbox{$\mathsurround=0pt #1$}}{}}

            \def\onepc{$^{\ast\ast}$} \def\fivepc{$^{\ast}$}
\def\tenpc{$^{\dag}$}
\def\legend{\multicolumn{4}{l}{\footnotesize{Significance levels
:\hspace{1em} $\dag$ : 10\% \hspace{1em}
$\ast$ : 5\% \hspace{1em} $\ast\ast$ : 1\% \normalsize}}}
\newcommand{\bs}[1]{\boldsymbol{#1}}        % ЖИРНЫЙ последующий символ имени матрицы или векторы
\newcommand{\bsA}{\boldsymbol{A}}

%\setstretch{1}                          % Межстрочный интервал
\flushbottom                            % Эта команда заставляет LaTeX чуть растягивать строки, чтобы получить идеально прямоугольную страницу
\righthyphenmin=2                       % Разрешение переноса двух и более символов
\pagestyle{plain}                       % Нумерация страниц снизу по центру.
%\settimeformat{hhmmsstime}              % Формат времени с секундами
\widowpenalty=300                       % Небольшое наказание за вдовствующую строку (одна строка абзаца на этой странице, остальное --- на следующей)
\clubpenalty=3000                       % Приличное наказание за сиротствующую строку (омерзительно висящая одинокая строка в начале страницы)
\setlength{\parindent}{1.5em}           % Красная строка.
%\embedfile[desc={Исходный код этого файла для LaTeX2e}]{\jobname.tex}                % Включение кода в выходной файл
\setlength{\topsep}{0pt}                % Уничтожение верхнего отступа, если он где проявится
\usepackage[pdftex,unicode,colorlinks=true,urlcolor=blue]{hyperref} % Ссылки в PDF

\renewcommand{\emptyset}{\varnothing}

\DeclareMathOperator{\E}{\mathbb{E}}
\DeclareMathOperator{\PP}{\mathbb{P}}
\DeclareMathOperator{\V}{\mathbb{V}}
\DeclareMathOperator{\CM}{\mathbb{C}}
\renewcommand{\C}{\CM}
\DeclareMathOperator{\var}{\mathrm{var}}
\DeclareMathOperator{\cov}{\mathrm{cov}}
\DeclareMathOperator{\MSE}{\mathrm{MSE}}
\DeclareMathOperator{\Bias}{\mathrm{Bias}}
\renewcommand{\P}{\PP}
\newcommand{\dsim}{\stackrel{d}{\sim}}
\newcommand{\hn}{\mathcal{H}_0}
\newcommand{\ha}{\mathcal{H}_a}
\newcommand{\x}{\bs x}
\newcommand{\y}{\bs y}
\newcommand{\thetab}{\bs \theta}
\newcommand{\e}{\varepsilon}
\newcommand{\pv}{\mathrm{P-value}}
\newcommand{\N}{\mathcal{N}}
\newcommand{\MLE}{\scriptscriptstyle MLE}
\newcommand{\LR}{\mathrm{LR}}
\newcommand{\I}{\mathbb{I}}
\begin{document} % начало документа
В данном разделе приводятся определения, и поясняется смысл основных макроэкономических переменных, о которых каждый день можно услышать немало информации в новостных сводках, обсуждениях в ток-шоу и политических спорах. Мы выделили 24 основные макроэкономические переменные, однако их количество зависит от того, какие переменные мы считаем ключевыми. В ходе определения этих ключевых величин, которыми макроэкономисты описывают нашу жизнь, и которые, в свою очередь, оказывают колоссальное влияние на каждого из нас, мы приводим статистические сводки и иллюстрируем смысл многих переменных кейсами, чтобы их значимость для каждого нас и их смысл были более прозрачны. 
После прочтения данного раздела Вы будете существенно более свободно ориентироваться в сводках экономических новостей, сможете лучше понимать суть и возможные последствия происходящих экономических событий. Однако для того, чтобы лучше понимать детали, стоящие за обсуждаемыми здесь показателями, а также их ограничения, необходимо ознакомиться с \textcolor{red}{соответствующими тематическими разделами в главе 1}. Для более глубокого понимания причинно-следственных взаимосвязи между макроэкономическими величинами, необходимо ознакомиться с вводными уровнями \textcolor{red}{соответствующих глав}.

Итак, преступим к определению и краткому обсуждению макроэкономических переменных. Дальнейший порядок представления переменных основан на следующей логике: сначала обсуждаются величины, которые не имеют привязки к конкретному \textcolor{red}{макроэкономическому агенту} (будь то домашние хозяйства, фирмы, государство или иностранный сектор), то есть наблюдаются, к примеру, даже в закрытой экономике (без иностранного сектора), или даже в экономике без государственного сектора; после чего представляются переменные, относящиеся к домашним хозяйствам и фирмам, затем к государственному сектору, после чего – к иностранному сектору.

	ВВП (номинальный) 

Определение. \bf{Номинальный ВВП} (или ВВП в текущих ценах; англ: Gross Domestic Product, GDP, current prices) – это совокупная рыночная стоимость всех конечных товаров и услуг, произведенных на территории данной страны за рассматриваемый промежуток времени.

Иными словами, номинальный ВВП представляет собой оценку стоимости всего объема производства в стране по текущим рыночным (за некоторыми исключениями) ценам.

Данная величина показывает размер экономики, ее производственную мощь. Однако номинальный ВВП может расти как за счет увеличения производства товаров и услуг, так и за счет роста уровня цен от года к году. Поэтому, чтобы устранить эффект роста цен на величину ВВП, макроэкономисты используют реальный ВВП.

	ВВП (реальный)

Определение. Реальный ВВП (или ВВП в базовых ценах; англ: Gross Domestic Product, constant prices) – это совокупная стоимость всех конечных товаров и услуг, произведенных в стране за рассматриваемый промежуток времени, измеренная в ценах базового года. 

Таким образом, реальный ВВП отражает оценку стоимости всех произведенных на территории страны товаров и услуг в ценах базового года, взятого за точку отсчета.

Реальный ВВП часто используется экономистами для оценок темпа роста экономики, поскольку этот показатель, в отличие от номинального ВВП, отчищен от влияния инфляции (роста цен). Однако выпуск может расти благодаря увеличению численности населения (чем больше людей живет в стране, тем больше людей может работать, тем больше выпуск), поэтому важно понимать не только то, на сколько велик ВВП страны в принципе, а сколько приходится на одного человека (поэтому смотрят на ВВП на душу населения). Другая проблема с реальным ВВП состоит в том, что стоимость товаров и услуг, даже в ценах на один год, может различаться между разными странами, поэтому для межстрановых сравнений применяют ВВП по паритету покупательной способности.

	ВВП (по паритету покупательной способности, ППС)

Определение. ВВП по паритету покупательной способности (англ: GDP Purchasing Power Parity (PPP)) – это оценка стоимости всех произведенных в стране товаров и услуг, рассчитанная в ценах некоторого государства. Применяется для сравнения ВВП в разных странах, поскольку помогает устранить различия в уровнях цен разных товаров и услуг в разных странах. Обычно за базовую страну для оценки этого показателя используют США.

Определение. Паритет покупательной способности – это такая стоимость одной валюты, выраженная в единицах другой, при которой один и тот же набор товаров и услуг в разных странах стоит одинаково.

Так, например, если в России на 80 руб. можно купить два литра молока, а в США на 1 доллар только 1 литр, причем рыночный курс составляет 70 р./долл., то ППС не выполнен, т.к. на один доллар в РФ и в США можно купить разное количество товаров.

Вставка с данными (например, данные по РФ, сопоставление с другими странами; динамика ВВП по ППС в РФ и т.п.)

	ВВП на душу населения 

Определение. ВВП на душу населения – это совокупный уровень выпуска (ВВП), который приходится на одного жителя страны. Таким образом, y=Y/L, где Y – ВВП, L – численность населения страны.

ВВП на душу населения, таким образом, показывает средний уровень доходов в стране. Данный показатель является одним из наиболее широко используемых для измерения уровня развития страны и уровня жизни ее граждан (ограничения этого показателя и альтернативные меры – см. раздел 1.4).
 
	Разрыв ВВП

Определение. Разрыв ВВП (англ: GDP gap) – это разница между текущим выпуском и потенциальным уровнем ВВП. Так, GDP_gap=Y_t-Y ̅, где Y_t – текущий уровень ВВП,  Y ̅ – потенциальный уровень ВВП.

Определение. Потенциальный уровень ВВП – это такой уровень выпуска в экономике, который достигается при полной возможной занятости ресурсов (капитала, труда, земли и др.). При этом полная занятость ресурсов не подразумевает полной занятости (уровень безработицы должен находиться на естественном уровне).

Разрыв выпуска может быть либо рецессионным (в случае, если текущий выпуск ниже потенциального уровня), либо инфляционным (в случае, если текущий выпуск выше потенциального уровня). Подробнее – см. раздел 1.9 Деловые циклы.

	Темп экономического роста 

Определение. Темп экономического роста (англ: Economic Growth Rate) – это отношение прироста ВВП к его значению на начало периода. Иначе говоря, на сколько процентов изменяется ВВП за определенный промежуток времени: g=(Y_1-Y_0)/Y_0 ⋅100% , где g – это темп экономического роста, Y_1 – выпуск на конец текущего периода, Y_0 – выпуск на конец предыдущего периода.

	Индекс цен (или уровень цен) 

– это некоторый обобщенный показатель, отражающий уровень цен в стране. 


	Темп инфляции
	Объем денежной массы
	Уровень безработицы
	Естественный уровень безработицы
	Ставка процента (пояснить, что они бывают разные)

	Потребительские расходы домашних хозяйств 
	Инвестиционные расходы фирм

	Государственные расходы
	Налоговые доходы государства (и трансферты)
	Первичный (и вторичный) дефицит государственного бюджета
	Объем государственного долга

	Валовый национальный доход (ВНД)
	Чистый экспорт (оно же – сальдо торгового баланса)
	Счет движения капитала
	Платежный баланс
	Валютный курс (номинальный)
	Валютный курс (реальный)
	Официальные резервы центрального банка



\end{document}
